\begin{blocksection}
\question Sketch the corresponding Fourier image for this function. For convenience, when drawing you can pretend that brightnesses are inverted (i.e.~draw black as white, and white as black). Assume that the peak-to-peak (white to white) distance is 64 pixels. Label significant points on the Fourier image with their frequencies.

\begin{figure}[H]
\centering
\includegraphics[width=0.25\textwidth]{spatial1}
\end{figure}

\vspace*{20mm}

\begin{solution}[0.75in]
{\color{red} TBA}
\end{solution}
\end{blocksection}

%%%

\begin{blocksection}
\question These are frequency representations of image filters. What does each filter do?

\begin{figure}[H]
\centering
\includegraphics[width=0.25\textwidth]{freq1} \hspace*{4mm} \includegraphics[width=0.25\textwidth]{freq2}
\end{figure}

\begin{solution}[0.75in]
{\color{red} TBA}
\end{solution}
\end{blocksection}
